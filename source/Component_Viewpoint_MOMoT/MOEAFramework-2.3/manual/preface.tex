% Copyright 2011-2014 David Hadka.  All Rights Reserved.
%
% This file is part of the MOEA Framework User Manual.
%
% Permission is granted to copy, distribute and/or modify this document under
% the terms of the GNU Free Documentation License, Version 1.3 or any later
% version published by the Free Software Foundation; with the Invariant Section
% being the section entitled "Preface", no Front-Cover Texts, and no Back-Cover
% Texts.  A copy of the license is included in the section entitled "GNU Free
% Documentation License".

\chapter*{Preface}

Thank you for your interest in the MOEA Framework.  Development of the MOEA Framework started in 2009 with the idea of providing a single, consistent framework for designing, testing, and experimenting with multiobjective evolutionary algorithms (MOEAs).  In late 2011, the software was open-sourced and made available to the general public.  Since then, a large user base from the MOEA research community has formed, with thousands of users from over 112 countries.  We are indebted to the many individuals who have contributed to this project through feedback, bug reporting, code development, and testing.

As of September 2013, we have reached the next major milestone in the development and maturity of the MOEA Framework.  Version 2.0 brings with it significant changes aimed to improve both the functionality and ease-of-use of this software.  We plan to implement more algorithms within the MOEA Framework, which will improve the reliability, performance, and flexibility of the algorithms.  Doing so places the responsibility of ensuring the correctness of the MOEA implementations on our shoulders, and we will continuously work to ensure results obtained using the MOEA Framework meet the standards of scientific rigor.

We also want to reach out to the community of researchers developing new, state-of-the-art MOEAs and ask that they consider providing reference implementations of their MOEAs within the MOEA Framework.  Doing so not only disseminates your work to a wide user base, but you can take advantage of the many resources and functionality already provided by the MOEA Framework.  Please contact \mailto{contribute@moeaframework.org} if we can assist in any way.

\chapter*{Citing the MOEA Framework}

Please include a citation to the MOEA Framework in any academic publications which used or are based on results from the MOEA Framework.  For example, you can use the following in-text citation:

\begin{quote}
  \textit{This study used the MOEA Framework, version %VERSION%, available from http://www.moeaframework.org/.}
\end{quote}

\noindent
You can also cite this user manual as follows:

\begin{quote}
  Hadka, D.  ``MOEA Framework User Manual.''  Version %VERSION%, http://www.moeaframework.org, %DATE%.
\end{quote}